



LAPORAN PROGRAM TITLE

LOCATION


TARIKH
9:00 AM - 5:00 PM


PERUNDING MUBARAK RESOURCES



LOKASI
LOCATION

ANJURAN
ORGANIZER

ISI KANDUNGAN


LAPORAN KURSUS FIQH USRAH DAERAH KUALA SELANGOR

MAKLUMAT PROGRAM





OBJEKTIF KURSUS



Setelah mengikuti modul,peserta akan:

TENTATIF PROGRAM












PENILAIAN PROGRAM




*Rujuk lampiran A: Borang Penilaian Peserta (BPK-11/JPIE)

CADANGAN UNTUK PENAMBAHBAIKAN KESELURUHAN PROGRAM

CADANGAN PERUNDING
Cadangan supaya peserta tidak menarik diri pada saat-saat akhir.
Pemakluman pada asnaf nama perunding yang akan mengendalikan kursus agar mereka tidak menganggap scammer.
Perlu arahan wajib kehadiran peserta daripada Jabatan Pengurusan Pendidikan Lembaga Zakat Selangor.
Makluman pembayaran elaun perlu dijelaskan kerana peserta sentiasa bertanyakan tentang kemasukan elaun kepada perunding selepas program.


CADANGAN ASNAF/PELAJAR (PESERTA)
Keseluruhan program sangat memuaskan dan memberi kesan kepada peserta.
Sekiranya program ini diteruskan di kemudian hari, sediakan pengangkutan kepada peserta yang tidak mempunyai pengangkutan.
Sediakan penginapan suapaya tidak berulang alik..
Diharapkan agar pihak zakat dapat mengadakan lagi kursus-kursus seperti ini.
Tiada kekurangan pada program tetapi mohon segerakan elaun kehadiran memandangkan terpaksa mengeluarkan kos pengangkutan.
Lauk tengahari yang disediakan pada hari pertama tidak memuaskan.
Maknanan kurang memuaskan.
Menambah lebih masa pada kursus.
Sejujurnya program anjuran zakat dengan kerjasama luar sangat bagus dan penyampaian dalam setiap slot sangat jelas dan terbaik.

RUMUSAN PENILAIAN
Penilaian dibuat berdasarkan kepada borang penilaian yang telah diisi oleh (30) orang peserta yang hadir pada hari kedua. Penilaian dibuat merangkumi ENAM (6) komponen iaitu kandungan sesi, alat bantu mengajar, keberkesanan penyampaian, fasilitator, persekitaran dan penilaian keseluruhan impak program kepada peserta. Skala penilaian dibahagikan kepada lima dengan nilaian seperti berikut:

Peratusan yang diperolehi mewakili pencapaian secara objektif berdasarkan hasil maklum balas dari peserta. Maklum balas ini dapat digunakan oleh pihak Perunding dan Penganjur dalam melaksanakan langkah pembaikan untuk program yang akan datang. Umumnya rata-rata peserta memberikan nilaian amat baik kepada penganjuran kursus dan amat bermanfaat kepada mereka. Secara keseluruhan penilaian yang dibuat adalah seperti graf berikut:



























Berdasarkan kepada penilaian (20) orang peserta memberikan penilaian CEMERLANG, dan (10) orang peserta memberi penilaian BAIK. Secara keseluruhannya penilaian penganjuran kursus ini mendapat penerimaan yang baik daripada peserta, walaupun ada kalangan mereka berhadapan dengan kekangan dan terpaksa menggantikan suami/ isteri atas faktor kerja yang tidak dapat dielakkan. Secara keseluruhannya peserta memberikan penilaian CEMERLANG, dan memberikan penilaian BAIK. Ini membuktikan penganjuran kursus ini memberikan manfaat kepada mereka disamping itu hampir keseluruhan peserta mengesyorkan kursus seumpama ini terus dianjurkan dari masa ke semasa.


ANALISA PRA DAN POST
PRA PENILAIAN PESERTA


POST PENILAIAN PESERTA


ANALISA BORANG PRA DAN POST

Seramai (32) orang peserta yang telah mengisi Penilaian Pre dan (30) orang Post kursus. Penilaian Pre dan Post adalah untuk menguji tahap kefahaman terhadap 7 modul yang dibicarakan oleh perunding sepanjang kursus selama 2 hari iaitu:

Tadabbur Al-Quran Surah an-Nahl ayat 72
Keluarga Anugerah Terindah Daripada Allah
Fiqh Darah Wanita; Haid, Nifas & Istihadah
Kecantikan wanita atau isteri dalam Islam
Perhiasan & Pergaulan Wanita Menurut Syarak
Ibadah Wanita; Solat
Ibadah Wanita; Puasa



Kesemua modul ini merangkumi Pembinaan Diri, Tanggungjawab ibu bapa dan tanggungjawab suami isteri Penilaian di lakukan pada dua peringkat iaitu pada awal kursus (Pra) dan pada akhir kursus (Post). Pencapaian peserta secara keseluruhan dapat dinilai dengan pencapaian skor individu sebelum dan selepas kursus serta peningkatan kefahaman setelah selesai kursus. Skor Keseluruhannya seperti berikut:


SKOR POST SETIAP INDIVIDU

Skor Post Individu menunjukkan skor yang dicapai oleh setiap peserta pada akhir kursus setelah mengikuti dan mendalami Fiqh Usrah selama 2 hari. Skor yang dicapai oleh setiap individu berbeza-beza, ada yang menunjukkan peningkatan, penurunan dan tiada perubahan/ peningkatan. Keputusan yang telah diperolehi adalah seperti rajah berikut:





RUMUSAN PENILAIAN PRA POST



Berdasarkan kepada carta penilaian Pra dan Post seperti yang dikemukakan di atas, 21 ORANG () peserta menyatakan berlakunya PENINGKATAN kefahaman mereka sebelum dan selepas kursus, manakala  ORANG () peserta mengukur tahap kefahaman mereka sebelum dan selepas kursus adalah sama TIADA PENINGKATAN, Walaubagaimana pun, terdapat 5 ORANG () peserta berlakunya PENURUNAN kemungkinan peserta kurang jelas berkaitan dengan Pre dan post penilaian. Jika dilihat kepada skor terdapat  () yang tidak melengkapkan penilaian post. Ini kerana, peserta tersebut tidak hadir pada hari kedua. Terdapat percanggahan skor yang dicapai dengan penilaian keseluruhan kursus, rata-rata peserta menyatakan kepentingan kursus ini dan dapat meningkatkan kefahaman mereka dalam konteks Fiqh Usrah.
Kesimpulannya, jika diukur dengan peningkatan kefahaman peserta, Fiqh Usrah telah memberi banyak manfaat kepada para peserta untuk memperbaiki diri dengan ilmu-ilmu hukum hakam berkaitan haid, nifas,konsep kecantikan dalam islam serta meningkatkan kualiti ibadah wanita dan nilai-nilai Islam dalam kehidupan berkeluarga demi mendapat ketaqwaan dan memupuk kesedaran tentang tanggungjawab dan peranan dalam kekeluargaan.

LAPORAN JEMPUTAN / KEHADIRAN PESERTA


JUMLAH JEMPUTAN	: 
JUMLAH KEHADIRAN	: 
JUMLAH TIDAK HADIR	: 

LAPORAN PENILAIAN ASNAF INDIVIDU

RUMUSAN PENILAIAN
Penilaian dibuat berdasarkan kepada penilaian yang telah diisi oleh () orang peserta yang hadir pada hari kedua. Penilaian dibuat merangkumi tiga () komponen iaitu kandungan sesi, keberkesanan penyampaian dan penilaian keseluruhan impak program kepada peserta. Maklum balas ini dapat digunakan oleh pihak Perunding dan Penganjur dalam melaksanakan langkah pembaikan untuk program yang akan datang. Umumnya rata-rata peserta memberikan nilaian amat baik kepada penganjuran kursus dan amat bermanfaat kepada mereka.

(RUJUK LAMPIRAN B: LAPORAN PENILAIAN INDIVIDU ASNAF SEPERTI DILAMPIRKAN)

GAMBAR PROGRAM








KESIMPULAN

Secara keseluruhannya penganjuran Kursus Fiqh Usrah secara bersemuka amat bersesuaian dan bertepatan dengan keperluan semasa. Pendedahan dan tips yang diberikan sepanjang kursus berlangsung boleh dijadikan sebagai panduan kepada mereka berhadapan dengan saat sukar ini. Langkah berani Lembaga Zakat Selangor (LZS) menganjurkan kursus membuktikan keprihatianan LZS dengan keperluan dan pentingnya kursus ini kepada para asnaf.



