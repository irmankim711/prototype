% Setting up the document class and basic configurations
\documentclass[a4paper,12pt]{article}
\usepackage[utf8]{inputenc}
\usepackage[T1]{fontenc}
\usepackage{geometry}
\geometry{margin=1in}
\usepackage{booktabs}
\usepackage{graphicx}
\usepackage{longtable}
\usepackage{colortbl}
\usepackage{xcolor}
\usepackage{enumitem}
\usepackage{amsmath}
\usepackage{pdflscape}
\usepackage{noto}

% Defining colors for tables
\definecolor{lightgray}{gray}{0.9}

% Document begins
\begin{document}

% Title Page
\begin{center}
    \vspace*{1cm}
    \textbf{\huge LAPORAN SAMPLE PROGRAM}\\
    \vspace{0.5cm}
    \textbfSAMPLE LOCATION\\
    \vspace{0.5cm}
    \textbf{Tarikh: 01/01/2025}\\
    \vspace{0.5cm}
    \textbf{Masa: 9:00 AM - 5:00 PM}\\
    \vspace{1cm}
    \textbf{ANJURAN: SAMPLE ORGANIZER}\\
    \vspace{0.5cm}
    \textbf{PERUNDING: MUBARAK RESOURCES}
\end{center}

% Table of Contents
\tableofcontents
\newpage

% Section 1: Program Information
\section{Maklumat Program}
\begin{tabular}{|l|l|}
    \hline
    \textbf{Latar Belakang Kursus} & SAMPLE BACKGROUND \\
    \hline
    \textbf{Tarikh} & 01/01/2025 \\
    \hline
    \textbf{Tempat} &  \\
    \hline
    \textbf{Penceramah} & SAMPLE SPEAKER \\
    \hline
    \textbf{Jurulatih} & SAMPLE TRAINER \\
    \hline
    \textbf{Fasilitator} & SAMPLE FACILITATOR \\
    \hline
    \textbf{Jumlah Peserta (Lelaki)} & 0 \\
    \hline
    \textbf{Jumlah Peserta (Perempuan)} & 0 \\
    \hline
    \textbf{Jumlah Keseluruhan Hadir} & 0 \\
    \hline
    \textbf{Urusetia} &  \\
    \hline
    \textbf{Syarikat Pengangkutan} &  \\
    \hline
    \textbf{Katering} &  \\
    \hline
\end{tabular}

% Section 2: Course Objectives
\section{Objektif Kursus}
Setelah mengikuti modul, peserta akan:
\begin{itemize}
    \item SAMPLE OBJECTIVES
\end{itemize}

% Section 3: Program Tentative
\section{Tentatif Program}
\subsection{Hari Pertama ()}
\begin{longtable}{|p{2cm}|p{3cm}|p{7cm}|p{3cm}|}
    \hline
    \rowcolor{lightgray}
    \textbf{Masa} & \textbf{Perkara/Aktiviti} & \textbf{Penerangan} & \textbf{Pengendali Slot} \\
    \hline
    \endhead
    
\end{longtable}

\subsection{Hari Kedua ()}
\begin{longtable}{|p{2cm}|p{3cm}|p{7cm}|p{3cm}|}
    \hline
    \rowcolor{lightgray}
    \textbf{Masa} & \textbf{Perkara/Aktiviti} & \textbf{Penerangan} & \textbf{Pengendali Slot} \\
    \hline
    \endhead
    
\end{longtable}

% Section 4: Program Evaluation
\section{Penilaian Program}
\begin{tabular}{|c|p{6cm}|c|c|c|c|c|}
    \hline
    \rowcolor{lightgray}
    \textbf{Jenis Penilaian} & \textbf{Skala} & \textbf{1} & \textbf{2} & \textbf{3} & \textbf{4} & \textbf{5} \\
    \hline
    \multicolumn{2}{|c|}{Skala Penilaian} & Tidak Memuaskan & Kurang Memuaskan & Memuaskan & Baik & Cemerlang \\
    \hline
    \multicolumn{7}{|c|}{\textbf{A – Kandungan Kursus}} \\
    \hline
    Menepati objektif kursus & Jumlah Penilaian &  &  &  &  &  \\
    \hline
    Kandungan modul memberi impak & Jumlah Penilaian &  &  &  &  &  \\
    \hline
    Jangka masa sesi & Jumlah Penilaian &  &  &  &  &  \\
    \hline
    \multicolumn{7}{|c|}{\textbf{B – Alat Bantu Mengajar}} \\
    \hline
    Cetakkan nota & Jumlah Penilaian &  &  &  &  &  \\
    \hline
    Nota mudah fahami & Jumlah Penilaian &  &  &  &  &  \\
    \hline
    Penggunaan ‘white board’ & Jumlah Penilaian &  &  &  &  &  \\
    \hline
    Sistem LCD & Jumlah Penilaian &  &  &  &  &  \\
    \hline
    Penggunaan PA sistem & Jumlah Penilaian &  &  &  &  &  \\
    \hline
    \multicolumn{7}{|c|}{\textbf{C – Keberkesanan Penyampai}} \\
    \hline
    Persediaan yang rapi & Jumlah Penilaian &  &  &  &  &  \\
    \hline
    Penyampaian yang teratur & Jumlah Penilaian &  &  &  &  &  \\
    \hline
    Bahasa mudah difahami & Jumlah Penilaian &  &  &  &  &  \\
    \hline
    Pengetahuan tajuk kursus & Jumlah Penilaian &  &  &  &  &  \\
    \hline
    Menjawab soalan dengan baik & Jumlah Penilaian &  &  &  &  &  \\
    \hline
    Metodologi latihan sesuai & Jumlah Penilaian &  &  &  &  &  \\
    \hline
    Menarik minat peserta & Jumlah Penilaian &  &  &  &  &  \\
    \hline
    Maklumbalas peserta & Jumlah Penilaian &  &  &  &  &  \\
    \hline
    \multicolumn{7}{|c|}{\textbf{D – Fasilitator}} \\
    \hline
    Memberi impak kepada peserta & Jumlah Penilaian &  &  &  &  &  \\
    \hline
    Prestasi keseluruhan & Jumlah Penilaian &  &  &  &  &  \\
    \hline
    \multicolumn{7}{|c|}{\textbf{E – Persekitaran}} \\
    \hline
    Lokasi kursus & Jumlah Penilaian &  &  &  &  &  \\
    \hline
    Kemudahan tempat ibadah & Jumlah Penilaian &  &  &  &  &  \\
    \hline
    Kemudahan asas & Jumlah Penilaian &  &  &  &  &  \\
    \hline
    Penyediaan makan/minuman & Jumlah Penilaian &  &  &  &  &  \\
    \hline
    Kemudahan dewan seminar & Jumlah Penilaian &  &  &  &  &  \\
    \hline
    \multicolumn{7}{|c|}{\textbf{F – Keseluruhan Kursus}} \\
    \hline
    Memberi impak kepada peserta & Jumlah Penilaian &  &  &  &  &  \\
    \hline
    Prestasi keseluruhan & Jumlah Penilaian &  &  &  &  &  \\
    \hline
\end{tabular}

% Section 5: Improvement Suggestions
\section{Cadangan Penambahbaikan}
\subsection{Cadangan Perunding}
\begin{itemize}
    \item []
\end{itemize}

\subsection{Cadangan Asnaf/Peserta}
\begin{itemize}
    \item []
\end{itemize}

% Section 6: Evaluation Summary
\section{Rumusan Penilaian}
Penilaian dibuat berdasarkan borang penilaian yang diisi oleh  orang peserta. Penilaian merangkumi enam komponen: kandungan sesi, alat bantu mengajar, keberkesanan penyampaian, fasilitator, persekitaran, dan impak keseluruhan program. Skala penilaian adalah seperti berikut:

\begin{tabular}{|c|c|c|c|c|}
    \hline
    \rowcolor{lightgray}
    \textbf{1} & \textbf{2} & \textbf{3} & \textbf{4} & \textbf{5} \\
    \hline
    Tidak Memuaskan & Kurang Memuaskan & Memuaskan & Baik & Cemerlang \\
    \hline
\end{tabular}

\begin{tabular}{|c|c|c|c|c|}
    \hline
    \rowcolor{lightgray}
    \textbf{Peratus} & \textbf{Tidak Memuaskan} & \textbf{Kurang Memuaskan} & \textbf{Memuaskan} & \textbf{Baik} & \textbf{Cemerlang} \\
    \hline
    &  &  &  &  &  \\
    \hline
\end{tabular}

% Placeholder for chart (to be generated by automation system)
% 

% Section 7: Pre and Post Analysis
\section{Analisa Pra dan Post}
\subsection{Pra Penilaian Peserta}
\begin{longtable}{|c|p{6cm}|c|}
    \hline
    \rowcolor{lightgray}
    \textbf{Bil} & \textbf{Nama} & \textbf{Markah Pra} \\
    \hline
    \endhead
    
\end{longtable}

\subsection{Post Penilaian Peserta}
\begin{longtable}{|c|p{6cm}|c|}
    \hline
    \rowcolor{lightgray}
    \textbf{Bil} & \textbf{Nama} & \textbf{Markah Post} \\
    \hline
    \endhead
    
\end{longtable}

\subsection{Analisa Pra dan Post}
\begin{longtable}{|c|p{6cm}|c|c|c|}
    \hline
    \rowcolor{lightgray}
    \textbf{Bil} & \textbf{Nama} & \textbf{Markah Pra} & \textbf{Markah Post} & \textbf{Kenaikan/Penurunan} \\
    \hline
    \endhead
    
\end{longtable}

\subsection{Rumusan Penilaian Pra Post}
\begin{tabular}{|c|c|c|c|c|}
    \hline
    \rowcolor{lightgray}
    \textbf{} & \textbf{Menurun} & \textbf{Tiada Peningkatan} & \textbf{Meningkat} & \textbf{Tidak Lengkap} \\
    \hline
    Peratus &  &  &  &  \\
    \hline
    Orang &  &  &  &  \\
    \hline
\end{tabular}

% Section 8: Attendance Report
\section{Laporan Kehadiran Peserta}
\begin{longtable}{|c|p{4cm}|p{3cm}|p{4cm}|p{2cm}|c|c|p{2cm}|}
    \hline
    \rowcolor{lightgray}
    \textbf{Bil} & \textbf{Nama} & \textbf{No. K/P} & \textbf{Alamat} & \textbf{No. Tel} & \textbf{Hadir H1} & \textbf{Hadir H2} & \textbf{Catatan} \\
    \hline
    \endhead
    
\end{longtable}

\textbf{Jumlah Jemputan:}  \\
\textbf{Jumlah Kehadiran:} 0 \\
\textbf{Jumlah Tidak Hadir:} 0

% Section 9: Individual Asnaf Evaluation
\section{Laporan Penilaian Asnaf Individu}
Penilaian dibuat berdasarkan  orang peserta yang hadir pada hari kedua. Penilaian merangkumi tiga komponen: kandungan sesi, keberkesanan penyampaian, dan impak keseluruhan program. Maklum balas ini boleh digunakan untuk penambahbaikan program akan datang.

% Section 10: Program Pictures
\section{Gambar Program}
\begin{figure}[h]
    \centering
    
\end{figure}

% Section 11: Conclusion
\section{Kesimpulan}


% Signatures
\begin{center}
    \vspace{1cm}
    \begin{tabular}{p{5cm}p{5cm}p{5cm}}
        \textbf{Disediakan Oleh:} & \textbf{Disemak Oleh:} & \textbf{Disahkan Oleh:} \\
        Perunding (MUBARAK RESOURCES) & Eksekutif Pembangunan Asnaf & Ketua Jabatan Pembangunan Insan dan Ekonomi \\
        CONSULTANT NAME & EXECUTIVE NAME & HEAD NAME \\
        Tarikh: 01/01/2025 & Tarikh: 01/01/2025 & Tarikh: 01/01/2025 \\
    \end{tabular}
\end{center}

\end{document}